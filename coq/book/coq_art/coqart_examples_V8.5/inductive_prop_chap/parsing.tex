\documentclass{report}
\title{Exercises on parsing}
\author{Yves Bertot, Pierre Cast\'eran}
\usepackage[latin1]{inputenc}
\usepackage[T1]{fontenc}
\usepackage{makeidx}
\usepackage{multind}
\usepackage{alltt}   
\usepackage{verbatim}
\usepackage{amssymb}
\usepackage{amsmath}
\usepackage{theorem}
\usepackage[dvips]{epsfig}
\usepackage{epic}
\usepackage{eepic}
\usepackage{ecltree}
\usepackage{moreverb} 
\usepackage{color}
\usepackage{pifont}
\usepackage{xr}
\usepackage{url}

\newcommand{\variantspringer}[1]{#1}
\newcommand{\marginok}[1]{\marginpar{\raggedright OK:#1}}
\newcommand{\tab}{{\null\hskip1cm}}
\newcommand{\Ltac}{\mbox{\emph{$\cal L$}tac}}
\newcommand{\coq}{\mbox{\emph{Coq}}}
\newcommand{\lcf}{\mbox{\emph{LCF}}}
\newcommand{\hol}{\mbox{\emph{HOL}}}
\newcommand{\pvs}{\mbox{\emph{PVS}}}
\newcommand{\isabelle}{\mbox{\emph{Isabelle}}}
\newcommand{\prolog}{\mbox{\emph{Prolog}}}
\newcommand{\goalbar}{\tt{}============================\it}
\newcommand{\gallina}{\mbox{\emph{Gallina}}}
\newcommand{\joker}{\texttt{\_}}
\newcommand{\eprime}{\(\e^{\prime}\)}
\newcommand{\Ztype}{\citecoq{Z}}
\newcommand{\propsort}{\citecoq{Prop}}
\newcommand{\setsort}{\citecoq{Set}}
\newcommand{\typesort}{\citecoq{Type}}
\newcommand{\ocaml}{\mbox{\emph{OCAML}}}
\newcommand{\haskell}{\mbox{\emph{Haskell}}}
\newcommand{\why}{\mbox{\emph{Why}}}
\newcommand{\Pascal}{\mbox{\emph{Pascal}}}

\newcommand{\ml}{\mbox{\emph{ML}}}

\newcommand{\scheme}{\mbox{\emph{Scheme}}}
\newcommand{\lisp}{\mbox{\emph{Lisp}}}

\newcommand{\implarrow}{\mbox{$\Rightarrow$}}
\newcommand{\metavar}[1]{?#1}
\newcommand{\notincoq}[1]{#1}
\newcommand{\coqscope}[1]{\%#1}
\newcommand{\arrow}{\mbox{$\rightarrow$}}
\newcommand{\fleche}{\arrow}
\newcommand{\funarrow}{\mbox{$\Rightarrow$}}
\newcommand{\ltacarrow}{\funarrow}
\newcommand{\coqand}{\mbox{\(\wedge\)}}
\newcommand{\coqor}{\mbox{\(\vee\)}}
\newcommand{\coqnot}{\mbox{\(\neg\)}}
\newcommand{\hide}[1]{}
\newcommand{\hidedots}[1]{...}
\newcommand{\sig}[3]{\texttt{\{}#1\texttt{:}#2 \texttt{|} #3\texttt{\}}}
\renewcommand{\neg}{\sim}
\renewcommand{\marginpar}[1]{}

\addtocounter{secnumdepth}{1}
\providecommand{\og}{�}
\providecommand{\fg}{�}
\definecolor{light}{rgb}{0.9,0.9,0.9}

\newcommand{\hard}{\mbox{\small *}}
\newcommand{\xhard}{\mbox{\small **}}
\newcommand{\xxhard}{\mbox{\small ***}}

%%% Operateurs, etc.
\newcommand{\impl}{\mbox{$\rightarrow$}}
\newcommand{\appli}[2]{\mbox{\tt{#1 #2}}}
\newcommand{\applis}[1]{\mbox{\texttt{#1}}}
\newcommand{\abst}[3]{\mbox{\tt{fun #1:#2 \funarrow #3}}}
\newcommand{\coqle}{\mbox{$\leq$}}
\newcommand{\coqge}{\mbox{$\geq$}}
\newcommand{\coqdiff}{\mbox{$\neq$}}
\newcommand{\coqiff}{\mbox{$\leftrightarrow$}}
\newcommand{\prodsym}{\mbox{\(\forall\,\)}}
\newcommand{\exsym}{\mbox{\(\exists\,\)}}

\newcommand{\substsign}{/}
\newcommand{\subst}[3]{\mbox{#1\{#2\substsign{}#3\}}}
\newcommand{\anoabst}[2]{\mbox{\tt[#1]#2}}
\newcommand{\letin}[3]{\mbox{\tt let #1:=#2 in #3}}
\newcommand{\prodep}[3]{\mbox{\tt \(\forall\,\)#1:#2,$\,$#3}}
\newcommand{\prodplus}[2]{\mbox{\tt\(\forall\,\)$\,$#1,$\,$#2}}
\newcommand{\dom}[1]{\textrm{dom}(#1)} % domaine d'un contexte (log function)
\newcommand{\norm}[1]{\textrm{n}(#1)} % forme normale (log function)
\newcommand{\coqZ}[1]{\mbox{\tt{`#1`}}}
\newcommand{\coqnat}[1]{\mbox{\tt{#1}}}
\newcommand{\coqcart}[2]{\mbox{\tt{#1*#2}}}
\newcommand{\alphacong}{\mbox{$\,\cong_{\alpha}\,$}} % alpha-congruence
\newcommand{\betareduc}{\mbox{$\,\rightsquigarrow_{\!\beta}$}\,} % beta reduction
%\newcommand{\betastar}{\mbox{$\,\Rightarrow_{\!\beta}^{*}\,$}} % beta reduction
\newcommand{\deltareduc}{\mbox{$\,\rightsquigarrow_{\!\delta}$}\,} % delta reduction
\newcommand{\dbreduc}{\mbox{$\,\rightsquigarrow_{\!\delta\beta}$}\,} % delta,beta reduction
\newcommand{\ireduc}{\mbox{$\,\rightsquigarrow_{\!\iota}$}\,} % delta,beta reduction


% jugement de typage
\newcommand{\these}{\boldsymbol{\large \vdash}}
\newcommand{\disj}{\mbox{$\backslash/$}}
\newcommand{\conj}{\mbox{$/\backslash$}}
%\newcommand{\juge}[3]{\mbox{$#1 \boldsymbol{\vdash} #2 : #3 $}}
\newcommand{\juge}[4]{\mbox{$#1,#2 \these #3 \boldsymbol{:} #4 $}}
\newcommand{\smalljuge}[3]{\mbox{$#1 \these #2 \boldsymbol{:} #3 $}}
\newcommand{\goal}[3]{\mbox{$#1,#2 \these^{\!\!\!?} #3  $}}
\newcommand{\sgoal}[2]{\mbox{$#1\these^{\!\!\!\!?} #2 $}}
\newcommand{\reduc}[5]{\mbox{$#1,#2 \these #3 \rhd_{#4}#5 $}}
\newcommand{\convert}[5]{\mbox{$#1,#2 \these #3 =_{#4}#5 $}}
\newcommand{\convorder}[5]{\mbox{$#1,#2 \these #3\leq _{#4}#5 $}}
\newcommand{\wouff}[2]{\mbox{$\emph{WF}(#1)[#2]$}}


%\newcommand{\mthese}{\underset{M}{\vdash}}
\newcommand{\mthese}{\boldsymbol{\vdash}_{\!\!M}}
\newcommand{\type}{\boldsymbol{:}}

% jugement absolu

%\newcommand{\ajuge}[2]{\mbox{$ \boldsymbol{\vdash} #1 : #2 $}}
\newcommand{\ajuge}[2]{\mbox{$\these #1 \boldsymbol{:} #2 $}}

%%% logique minimale
\newcommand{\propzero}{\mbox{$P_0$}} % types de Fzero

%%% logique propositionnelle classique
\newcommand {\ff}{\boldsymbol{f}} % faux
\newcommand {\vv}{\boldsymbol{t}} % vrai

\newcommand{\verite}{\mbox{$\cal{B}$}} % {\ff,\vv}
\newcommand{\sequ}[2]{\mbox{$#1 \vdash #2 $}} % sequent
\newcommand{\strip}[1]{#1^o} % enlever les variables d'un contexte



%%% tactiques
\newcommand{\decomp}{\delta} % decomposition
\newcommand{\recomp}{\rho} % recomposition

%%% divers
\newcommand{\cqfd}{\mbox{\textbf{cqfd}}}
\newcommand{\fail}{\mbox{\textbf{F}}}
\newcommand{\succes}{\mbox{$\blacksquare$}}
%%% Environnements


%% Fzero
\newcommand{\con}{\mbox{$\cal C$}}
\newcommand{\var}{\mbox{$\cal V$}}

\newcommand{\atomzero}{\mbox{${\cal A}_0$}} % types de base de Fzero
\newcommand{\typezero}{\mbox{${\cal T}_0$}} % types de Fzero
\newcommand{\termzero}{\mbox{$\Lambda_0$}} % termes de Fzero 
\newcommand{\conzero}{\mbox{$\cal C_0$}} % contextes de Fzero 

\newcommand{\buts}{\mbox{$\cal B$}} % buts

%%% for drawing terms
% abstraction [x:t]e
\newcommand{\PicAbst}[3]{\begin{bundle}{\bf abst}\chunk{#1}\chunk{#2}\chunk{#3}%
                        \end{bundle}}

% the same in DeBruijn form
\newcommand{\PicDbj}[2]{\begin{bundle}{\bf abst}\chunk{#1}\chunk{#2}
                       \end{bundle}}


% applications
\newcommand{\PicAppl}[2]{\begin{bundle}{\bf appl}\chunk{#1}\chunk{#2}%
                         \end{bundle}}

% variables
\newcommand{\PicVar}[1]{\begin{bundle}{\bf var}\chunk{#1}
                          \end{bundle}}

% constantes
\newcommand{\PicCon}[1]{\begin{bundle}{\bf const}\chunk{#1}\end{bundle}}

% arrows
\newcommand{\PicImpl}[2]{\begin{bundle}{\impl}\chunk{#1}\chunk{#2}%
                         \end{bundle}}



%%%% scripts coq
\newcommand{\prompt}{\mbox{\sl Coq $<\;$}}
\newcommand{\natquicksort}{\texttt{nat\_quicksort}}
\newcommand{\citecoq}[1]{\mbox{\texttt{#1}}}
\newcommand{\safeit}{\it}
\newtheorem{exercice}{Exercise}
%\newtheorem{definition}{Definition}[chapter]
\begin{document}
\maketitle

This file presents a series of exercices about parsing  parenthesized 
expression (from Chap. 8 of the book).



\begin{exercice}\label{exo-wp-1}%solution inductive-prop-chap/SRC/parsing.v
This exercise starts a long series of exercises on parenthesized
expressions.  This series will finish with the construction of a
parser for well-parenthesized expressions; in other words, a program
that constructs a piece of data that represents the structure of an
expression (see Exercises~\ref{exo-wp-2} and~\ref{exo-wp-6}, 
pages~\pageref{exo-wp-2},~\pageref{exo-wp-6}).

We consider the following type of characters:
%V8 A prendre
%V8 A prendre
%A prendre
% inclusion numero 511
% traduction numero 1068
\begin{alltt}
Inductive par : Set := open | close.
\end{alltt}
We represent character strings using the type
``\citecoq{\appli{list}{par}}.''  An expression is well-parenthesized 
when:
\begin{enumerate}
\item it is the empty list,
\item it is a well-parenthesized expression between parentheses,
\item it is the concatenation of two well-parenthesized expressions.
\end{enumerate}
Define the inductive property
``\citecoq{wp:\appli{list}{par}\arrow{}Prop}'' that corresponds to this
informal definition.  You can use the function \citecoq{app} given in
the module \citecoq{List} to concatenate two lists.  Prove the
following two properties:
%V8 A prendre
% inclusion numero 513
% traduction numero 1072
\begin{alltt}
\hide{Check (fun (wp : list par \arrow Prop)(}wp_oc : wp (cons open (cons close nil))\hide{)(}

wp_o_head_c :
     {\prodsym}l1 l2:list par,
       wp l1 \arrow{} wp l2 \arrow{} wp (cons open (app l1 (cons close l2)))
\hide{)(}
wp_o_tail_c :
     {\prodsym}l1 l2:list par, wp l1 \arrow{} wp l2 \arrow{}
       wp (app l1 (cons open (app l2 (cons close nil))))\hide{)\funarrow False)}.
\end{alltt}
\end{exercice}
\begin{exercice}%solution inductive-prop-chap/SRC/parsing.v
\label{exo-wp-1-1}
This exercise continues Exercise~\ref{exo-wp-1}.  We consider a type of
binary trees without labels and a function that maps any tree to a
list of characters.  Show that this function always builds a
well-parenthesized expression:
\index{Examples}{Function definitions!bin\_to\_string}
%A prendre
%V8 A prendre
%V8 A prendre
% inclusion numero 514
% traduction numero 1074
\begin{alltt}
Inductive bin : Set := L : bin | N : bin\arrow{}bin\arrow{}bin.

Fixpoint bin_to_string (t:bin) : list par :=
  match t with
  | L \funarrow{} nil
  | N u v \funarrow{}
    cons open 
      (app (bin_to_string u)(cons close (bin_to_string v)))
  end.
\end{alltt}
\end{exercice}
\begin{exercice}\label{exo-wp-1-2}%solution inductive-prop-chap/SRC/parsing.v
This exercise continues Exercise~\ref{exo-wp-1-1}.
Prove that the following function also returns a well-parenthesized
expression:
\index{Examples}{Function definitions!bin\_to\_string'}
%A prendre
%V8 A prendre
%V8 A prendre
% inclusion numero 515
% traduction numero 1076
\begin{alltt}
Fixpoint bin_to_string' (t:bin) : list par :=
  match t with
  | L \funarrow{} nil
  | N u v \funarrow{}
    app (bin_to_string' u)
        (cons open (app (bin_to_string' v)(cons close nil)))
  end.
\end{alltt}
\end{exercice}



\begin{exercice}\label{exo-wp-2}{\hard} %solution inductive-prop-chap/SRC/parsing.v
This exercise continues Exercise~\ref{exo-wp-1},
page~\pageref{exo-wp-1}.  Here is a second definition of
well-parenthesized expressions.  Prove that it is equivalent to the
previous one:
%A prendre
%V8 A prendre
% inclusion numero 587
% traduction numero 1161
\begin{alltt}
Inductive wp' : list par \arrow{} Prop :=
| wp'_nil : wp' nil
| wp'_cons : {\prodsym}l1 l2:list par, wp' l1 \arrow{} wp' l2 \arrow{} 
                  wp' (cons open (app l1 (cons close l2))).
\end{alltt}
\end{exercice}
\begin{exercice}\label{exo-wp-3} {\hard} %solution inductive-prop-chap/SRC/parsing.v
This exercise continues Exercise~\ref{exo-wp-2}.  Here is a third
definition. Prove that it is equivalent to the previous ones:
%A prendre
%V8 A prendre
% inclusion numero 588
% traduction numero 1162
\begin{alltt}
Inductive wp'' : list par \arrow{} Prop :=
| wp''_nil : wp'' nil
| wp''_cons :
  {\prodsym}l1 l2:list par, wp'' l1 \arrow{} wp'' l2 \arrow{}
      wp'' (app l1 (cons open (app l2 (cons close nil)))).
\end{alltt}
\end{exercice}
\begin{exercice}\label{exo-wp-4}{\xhard} %solution inductive-prop-chap/SRC/parsing.v
This exercise continues Exercise~\ref{exo-wp-3}.  Here is a function that 
recognizes well-parenthesized expressions by counting the opening
parentheses that are not yet closed:
\index{Examples}{Function definitions!recognize}
%A prendre
%V8 A prendre
% inclusion numero 589
% traduction numero 1163
\begin{alltt}
Fixpoint recognize (n:nat)(l:list par)\{struct l\} : bool :=
  match l with
    nil \funarrow{} match n with O \funarrow{} true | _ \funarrow{} false end
  | cons open l' \funarrow{} recognize (S n) l'
  | cons close l' \funarrow{}
    match n with O \funarrow{} false | S n' \funarrow{} recognize n' l' end
  end.
\end{alltt}
Prove the following theorem:
\iffalse
%A prendre
%V8 A prendre
% inclusion numero 590
% traduction numero 1164
\begin{alltt}
(* This fragment is not in the book *)
Variable hypothesis wp : list par \arrow{} Prop.
Hypothesis
(* end of fragment *)
\end{alltt}
\fi
%A prendre
%V8 A prendre
% inclusion numero 591
% traduction numero 1165
\begin{alltt}
recognize_complete_aux :
  {\prodsym}l:list par,\, wp l \arrow{}
  {\prodsym}(n:nat)(l':list par), 
  recognize n (app l l') = recognize n l'.
\end{alltt}
Conclude with the following main theorem:
\iffalse
%A prendre
%V8 A prendre
% inclusion numero 592
% traduction numero 1166
\begin{alltt}
(* This fragment is not in the book *)
Hypothesis
(* end of fragment *)
\end{alltt}
\fi
%A prendre
%V8 A prendre
% inclusion numero 593
% traduction numero 1167
\begin{alltt}
recognize_complete :
  {\prodsym}l:list par, wp l \arrow{} recognize 0 l = true.
\end{alltt}
\end{exercice}
\begin{exercice}\label{exo-wp-5}{\xxhard} %solution inductive-prop-chap/SRC/parsing.v
This exercise is rather hard and continues Exercise~\ref{exo-wp-4}.
Prove that the \citecoq{recognize} function only accepts
well-parenthesized expressions, More precisely
\iffalse
%A prendre
%V8 A prendre
% inclusion numero 594
% traduction numero 1168
\begin{alltt}
(* This fragment is not in the book *)
Hypothesis
(* end of fragment *)
\end{alltt}
\fi
%A prendre
%V8 A prendre
% inclusion numero 595
% traduction numero 1169
\begin{alltt}
recognize_sound : {\prodsym}l:list par, recognize 0 l = true \arrow{} wp l.
\end{alltt}
Hint: we suggest proving that if ``\citecoq{\applis{recognize \(n\)
\(l\)}}'' is \citecoq{true}, then the string ``\citecoq{\applis{app
\(l_n\) \(l\)}}'' is well-parenthesized, where \(l_n\) is the string
made of \(n\) opening parentheses.  Several lemmas about list
concatenation are needed.
\end{exercice}
\begin{exercice}\label{exo-wp-7}{\xxhard} %solution inductive-prop-chap/SRC/parsing.v
This exercise continues Exercises~\ref{exo-wp-1-2}
and~\ref{exo-wp-3}.  We consider the following parsing function:
\index{Examples}{Function definitions!parse}
%A prendre
%V8 A prendre
% inclusion numero 596
% traduction numero 1170
\begin{alltt}
Fixpoint parse (s:list bin)(t:bin)(l:list par)\{struct l\}
 : option bin :=
  match l with
  | nil \funarrow{} match s with nil \funarrow{} Some t | _ \funarrow{} None end
  | cons open l' \funarrow{} parse (cons t s) L l'
  | cons close l' \funarrow{}
    match s with
    | cons t' s' \funarrow{} parse s' (N t' t) l'
    | _ \funarrow{} None
    end
  end.
\end{alltt}
Prove that this parser is correct and complete:
%A prendre
%V8 A prendre
% inclusion numero 598
% traduction numero 1171
\begin{alltt}
\hide{Hypothesis }parse_complete :
 {\prodsym}l:list par, wp l \arrow{} parse nil L l {\coqdiff} None.

\hide{Hypothesis }parse_invert:
  {\prodsym}(l:list par)(t:bin),
      parse nil L l = Some t \arrow{} bin_to_string' t = l.

\hide{Hypothesis }parse_sound:
 {\prodsym}(l:list par)(t:bin), parse nil L l = Some t \arrow{} wp l.
\end{alltt}
Hint: the completeness proof is easier when using the structure
proposed by the inductive definition \citecoq{wp'}. 
\end{exercice}


\begin{exercice}\label{exo-wp-6}{\xxhard} %solution inductive-prop-chap/SRC/parsing.v
This exercise continues Exercise~\ref{exo-wp-1-2}
page~\pageref{exo-wp-1-2}.  The following inductive definition gives
the description of a parsing function for well-paren\-the\-sized
expressions.  Intuitively, ``\citecoq{parse\_rel \(l_1\) \(l_2\) \(t\)}'' 
reads as ``parsing the string \(l_1\) leaves \(l_2\) as suffix
and builds the tree \(t\).''
%A prendre
%V8 A prendre
% inclusion numero 634
% traduction numero 1206
\begin{alltt}
Inductive parse_rel : list par \arrow{} list par \arrow{} bin \arrow{} Prop :=
  | parse_node :
    {\prodsym}(l1 l2 l3:list par)(t1 t2:bin),
      parse_rel l1 (cons close l2) t1 \arrow{} parse_rel l2 l3 t2 \arrow{} 
      parse_rel (cons open l1) l3 (N t1 t2)
  | parse_leaf_nil : parse_rel nil nil L
  | parse_leaf_close :
      {\prodsym}l:list par, parse_rel (cons close l)(cons close l) L.
\end{alltt}
Prove the following lemmas:
\iffalse
%A prendre
%V8 A prendre
% inclusion numero 635
% traduction numero 1207
\begin{alltt}
(* This fragment is not in the book *)
Hypothesis
(* end of fragment *)
\end{alltt}
\fi
%A prendre
%V8 A prendre
% inclusion numero 636
% traduction numero 1208
\begin{alltt}
parse_rel_sound_aux :
    {\prodsym}(l1 l2:list par)(t:bin),
      parse_rel l1 l2 t \arrow{} l1 = app (bin_to_string t) l2.
\end{alltt}
\iffalse
%A prendre
%V8 A prendre
% inclusion numero 637
% traduction numero 1209
\begin{alltt}
(* This fragment is not in the book *)
Hypothesis
(* end of fragment *)
\end{alltt}
\fi
%A prendre
%V8 A prendre
% inclusion numero 638
% traduction numero 1210
\begin{alltt}
parse_rel_sound :
    {\prodsym}l:list par, ({\exsym}t:bin, parse_rel l nil t)\arrow{} wp l.
\end{alltt}
\end{exercice}

\end{document}
