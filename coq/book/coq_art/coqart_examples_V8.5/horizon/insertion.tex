\documentclass{article}
\title {Comments on the sort program}
\author{Yves Bertot and Pierre Cast\'eran}
\usepackage[latin1]{inputenc}
\usepackage[T1]{fontenc}
%\usepackage{isolatin1}
\usepackage{makeidx}
\usepackage{alltt}
\usepackage{verbatim}
\usepackage{amssymb}
\usepackage{amsmath}
\usepackage{theorem}
\usepackage{t1enc}
\usepackage[dvips]{epsfig}
\usepackage{epic}
\usepackage{eepic}
\usepackage{ecltree}
\usepackage{moreverb} 
\usepackage{color}
\usepackage{pifont}
\usepackage{xr}
\usepackage{url}

\newcommand{\ifocho}{\iffalse}
\newcommand{\ifsiete}{\iftrue}

\newcommand{\appli}[2]{\mbox{\tt{(#1 #2)}}}
\newcommand{\applis}[1]{\mbox{\texttt{(#1)}}}
\newcommand{\abst}[3]{\mbox{\tt[#1:#2]#3}}
\newcommand{\prodep}[3]{\mbox{\tt(#1:#2)#3}}

\newcommand{\letin}[3]{\mbox{\tt[#1:=#2]#3}}

\newcommand{\coqversion}{}

\newcommand{\tab}{{\null\hskip1cm}}
\newcommand{\Ltac}{\mbox{\emph{$\cal L$}tac}}
\newcommand{\coq}{\mbox{\emph{Coq}}}
\newcommand{\lcf}{\mbox{\emph{LCF}}}
\newcommand{\hol}{\mbox{\emph{HOL}}}
\newcommand{\pvs}{\mbox{\emph{PVS}}}
\newcommand{\isabelle}{\mbox{\emph{Isabelle}}}
\newcommand{\prolog}{\mbox{\emph{Prolog}}}
\newcommand{\goalbar}{{\tt{}============================}}
\newcommand{\gallina}{\mbox{\emph{Gallina}}}
\newcommand{\ocaml}{\mbox{\emph{OCAML}}}
\newcommand{\haskell}{\mbox{\emph{Haskell}}}

\newcommand{\ml}{\mbox{\emph{ML}}}

\newcommand{\scheme}{\mbox{\emph{Scheme}}}
\newcommand{\lisp}{\mbox{\emph{Lisp}}}

\newcommand{\fleche}{\mbox{$\rightarrow$}}
\newcommand{\coqnot}{\mbox{\(\neg\)}}
% si Yves B. OK.  
\newcommand{\coqand}{\mbox{\(\wedge\)}}
\newcommand{\coqor}{\mbox{\(\vee\)}}
%\newcommand{\coqor}{\disj}
%\newcommand{\coqand}{\conj}

%modifications par PC
%\renewcommand{\Longrightarrow}{\fleche}
%\renewcommand{\Rightarrow}{\fleche}
\renewcommand{\neg}{\sim}

\addtocounter{secnumdepth}{1}
\providecommand{\og}{�}
\providecommand{\fg}{�}


%%% Environnements


\newcommand{\Fzero}{\mbox{${\cal F}_0$}}


\definecolor{setcolor}{rgb}{0.2,0,0}
\newcommand{\citeset}[1]{#1}
\newcommand{\variant}[1]{}
\newcommand{\hide}[1]{}
\newcommand{\hidedots}[1]{}
\definecolor{propcolor}{rgb}{0,0,0.2}
\newcommand{\citeprop}[1]{#1}
\definecolor{typecolor}{rgb}{0,0,0}
\newcommand{\citeterm}[1]{\textcolor{typecolor}{#1}}

\newcommand{\citecoq}[1]{\mbox{\texttt{#1}}}
\newcommand{\citetype}[1]{\citeterm{#1}}

\definecolor{light}{rgb}{0.9,0.9,0.9}

\newcommand{\meta}[1]{\mbox{$\langle$\textit{#1}$\rangle$}}
%%% Commandes pour le developpement


\newcommand{\brouillon}{\emph{(Brouillon du \today)}}
\newcommand{\ici}{\textbf{Je suis ici : \today}}

\newenvironment{agenda}{\em � faire :\begin{itemize}}{\end{itemize}}
\newenvironment{coqtt}{\noindent}{\noindent}
%%% Operateurs, etc.

\newcommand{\impl}{\mbox{$\rightarrow$}}


\newcommand{\sumbool}[2]{\mbox{\tt\{#1\}+\{#2\}}}

\newcommand{\subst}[3]{\mbox{#1\{#2/#3\}}}

\newcommand{\prodplus}[2]{\mbox{\tt(#1)#2}}
\newcommand{\dom}[1]{\textrm{dom}(#1)} % domaine d'un contexte (log function)
\newcommand{\norm}[1]{\textrm{n}(#1)} % forme normale (log function)
\newcommand{\coqnat}[1]{\mbox{\tt{(#1)}}}
\newcommand{\coqz}[1]{\mbox{\tt{`#1`}}}
\newcommand{\coqcart}[2]{\mbox{\tt{#1*#2}}}
\newcommand{\incoqcomment}[1]{{\tt (* #1 *)}}


\newcommand{\alphacong}{\mbox{$\,\cong_{\!\alpha}\,$}} % alpha-congruence
\newcommand{\betareduc}{\mbox{$\,\rightsquigarrow_{\!\beta}$}\,} % beta reduction
%\newcommand{\betastar}{\mbox{$\,\Rightarrow_{\!\beta}^{*}\,$}} % beta reduction
\newcommand{\deltareduc}{\mbox{$\,\rightsquigarrow_{\!\delta}$}\,} % delta reduction
\newcommand{\dbreduc}{\mbox{$\,\rightsquigarrow_{\!\delta\beta}$}\,} % delta,beta reduction
\newcommand{\ireduc}{\mbox{$\,\rightsquigarrow_{\!\iota}$}\,} % delta,beta reduction


% jugement de typage
\newcommand{\these}{\boldsymbol{\large \vdash}}
\newcommand{\disj}{\mbox{$\backslash/$}}
\newcommand{\conj}{\mbox{$/\backslash$}}
%\newcommand{\juge}[3]{\mbox{$#1 \boldsymbol{\vdash} #2 : #3 $}}
\newcommand{\juge}[4]{\mbox{$#1,#2 \these #3 \boldsymbol{:} #4 $}}
\newcommand{\smalljuge}[3]{\mbox{$#1 \these #2 \boldsymbol{:} #3 $}}
\newcommand{\goal}[3]{\mbox{$#1,#2 \these^{\!\!\!?} #3  $}}
\newcommand{\sgoal}[2]{\mbox{$#1\these^{\!\!\!\!?} #2 $}}
\newcommand{\reduc}[5]{\mbox{$#1,#2 \these #3 \rhd_{#4}#5 $}}
\newcommand{\convert}[5]{\mbox{$#1,#2 \these #3 =_{#4}#5 $}}
\newcommand{\convorder}[5]{\mbox{$#1,#2 \these #3\leq _{#4}#5 $}}
\newcommand{\wouff}[2]{\mbox{$\emph{WF}(#1)[#2]$}}


%\newcommand{\mthese}{\underset{M}{\vdash}}
\newcommand{\mthese}{\boldsymbol{\vdash}_{\!\!M}}
\newcommand{\type}{\boldsymbol{:}}

% jugement absolu

%\newcommand{\ajuge}[2]{\mbox{$ \boldsymbol{\vdash} #1 : #2 $}}
\newcommand{\ajuge}[2]{\mbox{$\these #1 \boldsymbol{:} #2 $}}

%%% logique minimale
\newcommand{\propzero}{\mbox{$P_0$}} % types de Fzero

%%% logique propositionnelle classique
\newcommand {\ff}{\boldsymbol{f}} % faux
\newcommand {\vv}{\boldsymbol{v}} % vrai

\newcommand{\verite}{\mbox{$\cal{B}$}} % {\ff,\vv}
\newcommand{\sequ}[2]{\mbox{$#1 \vdash #2 $}} % sequent
\newcommand{\strip}[1]{#1^o} % enlever les variables d'un contexte



%%% tactiques
\newcommand{\decomp}{\delta} % decomposition
\newcommand{\recomp}{\rho} % recomposition

%%% divers
\newcommand{\cqfd}{\mbox{\textbf{cqfd}}}
\newcommand{\fail}{\mbox{\textbf{F}}}
\newcommand{\succes}{\mbox{$\blacksquare$}}
%%% Environnements


%% Fzero
\newcommand{\con}{\mbox{$\cal C$}}
\newcommand{\var}{\mbox{$\cal V$}}

\newcommand{\atomzero}{\mbox{${\cal A}_0$}} % types de base de Fzero
\newcommand{\typezero}{\mbox{${\cal T}_0$}} % types de Fzero
\newcommand{\termzero}{\mbox{$\Lambda_0$}} % termes de Fzero 
\newcommand{\conzero}{\mbox{$\cal C_0$}} % contextes de Fzero 

\newcommand{\buts}{\mbox{$\cal B$}} % buts

%%% for drawing terms
% abstraction [x:t]e
\newcommand{\PicAbst}[3]{\begin{bundle}{\bf abst}\chunk{#1}\chunk{#2}\chunk{#3}%
                        \end{bundle}}

% the same in DeBruijn form
\newcommand{\PicDbj}[2]{\begin{bundle}{\bf abst}\chunk{#1}\chunk{#2}
                       \end{bundle}}


% applications
\newcommand{\PicAppl}[2]{\begin{bundle}{\bf appl}\chunk{#1}\chunk{#2}%
                         \end{bundle}}

% variables
\newcommand{\PicVar}[1]{\begin{bundle}{\bf var}\chunk{#1}
                          \end{bundle}}

% constantes
\newcommand{\PicCon}[1]{\begin{bundle}{\bf const}\chunk{#1}\end{bundle}}

% fleches
\newcommand{\PicImpl}[2]{\begin{bundle}{\impl}\chunk{#1}\chunk{#2}%
                         \end{bundle}}



%%%% scripts coq
\newcommand{\prompt}{\mbox{\sl Coq $<\;$}}


\begin{document}

\maketitle
% jugement de typage
We give some comments on a little developement in {\coq}~: a
certified sort program on lists of integers.

\section{Importing modules}

Our program will work on lists of integers. The modules \texttt{PolyList}
and \citecoq{ZArith} of {\coq}'s standard library 
contain the definition and some functions and already proved theorems 
on (respectively) polymorphic lists and integers.


\begin{verbatim}
Require PolyList.
Require ZArith.
\end{verbatim}

The type of lists of integers is now \appli{list}{Z}.

\section{List equivalence}
A sorting function working on lists must return a list which has
the same elements (with the same order of multiplicity) as its argument.
\subsection{Counting occurrences in a list}

In order to specify this constraint, we first define a function
which counts the number of occurences of some integer $z$ in a list
$l$. This function is defined recursively and uses a comparison function
for integers named \texttt{Z\_eq\_dec} (which we found in module
\texttt{ZArith}).

\begin{verbatim}
Fixpoint nb_occurences[z:Z; l:(list Z)]:nat :=
  Cases l of
    nil => O
  | (cons z' l') =>
     Cases (Z_eq_dec z z') of
       (left _) => (S (nb_occurences z l'))
     | (right _) => (nb_occurences z l')
     end
  end.
\end{verbatim}

We can test this function, for instance by computing the number of
occurences of $3$ in the list $(3,7,3)$. Notice we use a \emph{Lisp}-like
notation of lists.
\begin{alltt}
Eval Compute in 
  (nb_occurences `3` (cons `3` (cons `7` (cons `3` (nil ?))))).
\it
  = (2)
     : nat
\end{alltt}

\subsection{An equivalence relation on lists}
We can now define a binary relationship on \appli{list}{Z}.
Lists $l$ and $l'$ are related if they have the same number of
occurrenes of $z$, for any $z$.

\begin{verbatim}
Definition equiv[l,l':(list Z)]:=
     (z:Z)(nb_occurences z l)=(nb_occurences z l').
\end{verbatim}

\subsection{Some proofs}

The correctness proof of our sorting functions relies on some
properties of the \texttt{equiv} relationship. These properties
are interactively proved within the system {\coq}.

Each lemma or theorem is announced with its name and statement.
One can find between the keywords \texttt{Proof} and \citecoq{Qed} 
a sequence of commands for guiding the proof with the help of
\emph{tactics}. Instead of reading passively  such \emph{proof scripts},
you should \emph{replay} them on your computer, and watch the
answers of the system.

\subsubsection{The empty list is equivalent to itself}

The following proof is very simple. In few words, one computes the number
of occurrences of any number $z$ in the empty list, and the statement we 
want to proves reduces to the trivial equality $0=0$.

\begin{verbatim}
Lemma equiv_nil : (equiv (nil ?) (nil ?)).             
Proof.
  Intro z;Simpl;Auto.
Qed.
\end{verbatim}

\subsubsection{The \texttt{equiv} relation is a congruence \emph{wrt} 
\texttt{cons}}

We have to prove that, if $l$ and $l'$ are equivalent, and if we insert
the same $z$ in front of $l$ and $l'$, the resulting lists are equivalent too.

The proof is quite simple. We count the number of occurrences of 
some $z'$ in both resulting lists.  We have to consider the cases
where $z=z'$ and $z\not=z'$.

The statement of theorem \citecoq{equiv\_cons} 
must be read as follows~:
\begin{quote}
For each $z$ of type \texttt{Z}, for each $l$ and $l'$ of type
\appli{list}{Z}, if $l$ is equivalent to $l'$, then
\applis{cons $z$ $l$} is equivalent to \applis{cons $z$ $l'$}.
\end{quote}

\begin{verbatim}
Lemma equiv_cons : (z:Z;l,l':(list Z)) (equiv l l')->
                                       (equiv (cons z l) (cons z l')).
Proof.
 Intros z l l' H z';Simpl;Auto.
 Case (Z_eq_dec z' z);Simpl;Auto.
Qed.
\end{verbatim}

\subsubsection{More properties of \texttt{equiv}}
The relationship  \emph{equiv} is reflexive and transitive.

Proofs of both statement are quite easy. Notice the proof of
reflexivity, which is done by induction on lists, and uses
the lemma \emph{equiv\_cons}.

\begin{verbatim}
Lemma equiv_refl : (l:(list Z))(equiv l l).
Proof.
 Induction l;Intro z;Simpl;Auto.
 Intros;Apply equiv_cons;Auto.
Qed.

Lemma equiv_trans : (l,l',l'':(list Z))(equiv l l')->
                                       (equiv l' l'')->
                                       (equiv l l'').
Proof.
 Intros l l' l'' H H0 z.
 Transitivity  (nb_occurrences z l').
 Apply H;Auto.
 Apply H0;Auto.
Qed.
\end{verbatim}

Finally, we prove that our equivalence relation is compatible
with the permutation of the two first items of a list.
\begin{verbatim}
Lemma equiv_perm : (a,b:Z; l,l':(list Z))
                    (equiv l l')->
                    (equiv (cons a (cons b l))
                                   (cons b (cons a l'))).
Proof.
 Intros a b l l' H z;Simpl.
 Case (Z_eq_dec z a);Case (Z_eq_dec z b);Simpl;Case (H z);Auto.
Qed.

\end{verbatim}

\subsubsection{Helping {\coq} to help yourself}

The following command adds the four lemmas 
\texttt{equiv\_nil}, \texttt{equiv\_cons}, \texttt{equiv\_refl}
and \texttt{equiv\_perm} to a base of theorems which can be used
by the \texttt{Auto} tactics.

For instance, the foolowing goal (\emph{i.e.} prove that the
lists $(a,b,c,d)$ and $(b,a,d,c)$ are always equivalents ) is
solved in one automatic interaction.


\begin{verbatim}
Lemma abcd : (a,b,c,d:Z)
  (equiv (cons a (cons b (cons c (cons d (nil ?)))))
         (cons b (cons a (cons d (cons c (nil ?)))))).
Proof.
 Auto.
Qed.
\end{verbatim}


\subsection{ordered lists}

 The following piece of code defines the predicate `` to be
sorted '' on lists of integers.
 There are wo base cases, corresponding to lists with $0$ or $1$ item.
The recursive case corresponds to lists of the form 
$z_1z_2.l$ . Such a list is sorted if $z_1\leq z_2$ and the list $z_2.l$
 is sorted.
Definitions like the following one are called \emph{inductive definitions}
and are close -- but much more powerful -- to \emph{Prolog}'s clauses.


\begin{verbatim}
Inductive sorted : (list Z)-> Prop :=
 sorted0 : (sorted (nil Z))
|sorted1 : (z:Z)(sorted (cons z (nil ?)))
|sorted2 : (z1,z2:Z;l:(list Z))`z1<=z2` -> (sorted (cons z2 l))->
                                           (sorted (cons z1 (cons z2 l))).
Hints Resolve sorted0 sorted1 sorted2.
\end{verbatim}

The following interaction shows how to get automatically
a proof that some list is already sorted.
Note that the command ``\texttt{Auto with zarith}'' loads the
hints from the base of theorems about arithmetics on integers.

\begin{verbatim}
Lemma sorted_1135 : (sorted 
    (cons `1` (cons `1` (cons `3` (cons `5`  (nil ?)))))).
Proof.
 Auto  with zarith.
Qed.
\end{verbatim}

\section{Inserting in a sorted list}
The kernel of an insertion sort is a function which inserts
some item at the right place in an already sorted list.

\subsection{An insertion function}

Let us define a function for inserting some integer $z$ in a list 
$l$. This function is defined by a simple recursion on $l$.

\begin{itemize}
\item If $l$ is empty, we return the list compsed of a single
    occurrence of $z$,
\item If $l$ is of the form $a.l'$,
            \begin{itemize}
             \item if $z\leq a$, then we return $za.l$,
             \item if $a<z$, we return $a\applis{insert z l'}$.

\end{itemize}
\end{itemize}

\begin{verbatim}
Fixpoint insert[z:Z;l:(list Z)]:(list Z) :=
  Cases l of
          nil => (cons z (nil ?))
         |(cons a l') => Cases (Z_le_gt_dec z a) of
                          (left _)=>(cons z (cons a l'))
                         |(right _) =>(cons a (insert z l'))
                        end
  end.
\end{verbatim}

\begin{alltt}
Eval Compute in (insert `4` (cons `2` (cons `5` (nil ?)))).\it
     = (cons `2` (cons `4` (cons `5` (nil Z))))
     : (list Z)
\end{alltt}

\subsection{Some properties of \texttt{insert}}

The function \texttt{insert} has some properties, which will be used
in the validation of our sorting program.

\subsubsection{}
 We first prove that \applis{insert $x$ $l$} has exactly the same
elements as $x.l$.

\begin{verbatim}
Lemma insert_equiv : (l:(list Z);x:Z)(equiv (cons x l) (insert x l)).
Proof.
 Induction l ; Simpl ; Auto.
 Intros a l0 Hrec x.
 Case (Z_le_gt_dec x a);Simpl;Auto.
 Intro ;  Apply equiv_trans with (cons a (cons x l0)); Auto.
Qed.
\end{verbatim}

\subsubsection{}
 The following lemma is a justification for using
 \texttt{insert} in a sort : if $l$ is sorted, then
   \applis{insert $x$ $l$} returns a sorted list.

\begin{verbatim}
Lemma insert_sorted : (l:(list Z);x:Z)(sorted l)->(sorted (insert x l)).
Proof.
 Induction 1 ;  Simpl; Auto.
 Intro z; Case (Z_le_gt_dec x z);Simpl ; Auto with zarith.
 Intros z1 z2 ;Case (Z_le_gt_dec x z2);Simpl;Intros.
 Case (Z_le_gt_dec x z1);Simpl;Auto with zarith.
 Case (Z_le_gt_dec x z1);Simpl;Auto with zarith.
Qed.
\end{verbatim}

\section{A certified sorting program}
\subsection{From specification to realization}

We are now able to define a sorting function.
Instead of defining it as a plain recursive function like \texttt{insert},
we choose to ensure that the function we define has all the required
properties of a sort.

The following definition of the constant \texttt{sort} specifies its type,
not as the type of any function from \appli{list}{Z} to itself, but as 
the type of any function taking as argument a list $l$, and
returning a list $l'$ \emph{which has the same elements as $l$ and is
sorted too}.

Because of the constraints in its specification, it is easier to
define \texttt{sort} interactively, in the same style as the
preceding proofs.


\begin{verbatim}
Definition sort : (l:(list Z)){l':(list Z)|(equiv l l')/\(sorted l')}.
 Intro l;Induction l. 
 Exists (nil Z);Split;Auto.
 Case Hrecl;Intros l' [H0 H1].
 Exists (insert a l').
 Split.
 Apply equiv_trans with (cons a l') ; Auto.
 Apply insert_equiv.
 Apply insert_sorted ; Auto.
Defined.
\end{verbatim}

\subsection{Extraction to \emph{ML} code}
The command \texttt{Extraction "insert" insert sort} writes an
{\ocaml} module containing code extracted from the development above.
This code can be compiled, then executed or called from another
{\ocaml} program.


The theoretical principles on which {\coq} has been built ensure us
that this program is conform with its specification.


\section{Conclusion}
This little example shows some of the main features of {\coq}~:
possibility of defining datatypes and programs, proving properties
of programs with a reasonable degree of automation, and deriving 
correct programs from their specification.


The reader will learn to develop and certify more complex programs,
including imperative programs, polymorphic functions, and even 
entire modules. \coq's field of application includes mathematics,
logic, and many other domains (look at the users contributons).

\end{document}
